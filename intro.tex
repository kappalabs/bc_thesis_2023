% ==================================================
% ==================================================
\chapter{Introduction}
\label{intro}
The cerebral visual system is a complex biological system that has been a subject of research for almost a century. The primary animal models to study visual cortex are macaques~\citep{felleman1991distributed, boussaoud1990pathways}, cat, ferret, and  mouse~\citep{marshel2011functional, beltramo2019collicular}, but human studies are also common~\citep{boynton1996linear, wandell2007visual}. The primary visual cortex (V1) is the first cortical stage of visual processing, responsible for receiving and processing visual information from the retina. The cortical activity generated by V1 neurons can provide insights into visual perception.

Recent advancements in spiking neural network models have enabled us to simulate the activity of neurons in V1 in response to visual stimuli~\citep{antolik2018comprehensive}. By generating V1 cortical activity patterns through these models, we can study how the visual system processes and interprets visual stimuli, and how this activity relates to the perception of the individual.

One potential application of these models is to develop visual prostheses for individuals with visual impairments, such as blindness. By using V1 cortical activity generated by a spiking neural network, we can potentially design stimulation protocols for the visual prosthetic system, to restore some degree of vision to these individuals by generating visual stimuli that mimic the activity in their visual cortex.

However, before we can proceed with such clinical applications we need to first understand the relationship between the cortical activity in V1 and visual perception, which is the goal of this thesis. Specifically, we aim to develop a model that can decode stimulus images from the cortical activity in V1. This will enable us to explore the relationship between V1 activity and visual perception and develop new insights into how the visual system processes and interprets visual information.

One of the challenges of developing a model that can decode stimulus images from cortical activity is the limited availability of biological data. Collecting a sufficient amount of biological data is not only time-consuming but also requires extensive resources and expertise. Therefore, it makes sense to use the spiking neural network model of the V1, developed by Antolík and colleagues~\citep{antolik2018comprehensive}, which can potentially generate an unlimited amount of data needed to train these models. After this step, the models could be evaluated on expensive real biological data.

We will explore the use of machine learning techniques, particularly neural networks, to generate stimulus images from the V1 cortical activity model. Neural networks have been extensively used in computer vision and image processing, and they have shown great potential for generating high-quality images. One of the most common approaches is to use convolutional neural networks (CNNs) for image generation, as they are capable of capturing complex image features and patterns~\citep{radford2015unsupervised, xu2018attngan}.

To train our decoding models, we will use standard loss functions such as L1, MSE, SSIM, or MSSSIM, as well as more advanced discriminator loss from GAN networks training~\citep{goodfellow2014generative, goodfellow2020generative}. We will also employ linear regression models to examine the relationship between the V1 cortical activity and the corresponding stimulus images. This approach has been successfully used in previous studies to decode visual information from cortical activity~\citep{kay2008identifying}.

%The objective of this research is to develop a method for generating high-quality stimulus images based on cortical activity by combining the V1 cortical activity model with machine learning techniques. The spiking model of V1 proposed by~\citep{antolik2018comprehensive} is utilized to generate the dataset for training our models. The integration of machine learning techniques with the V1 cortical activity model is expected to result in a more accurate and efficient method for reconstructing visual stimuli.

\subsection*{Prior work}

In this thesis, we review several approaches for decoding visual information from neural activity. Retinal ganglion cell spikes have been used as a decoding tool by many researchers, including~\citep{zhang2020reconstruction}. We utilize their model as our second baseline along with the linear regression, which is a linear decoder, that is proposed by a few other following papers~\citep{Parthasarathy153759, kay2008identifying, kim2021nonlinear}.
Similar to the previous, in this paper~\citep{Parthasarathy153759}, they employed a linear decoder in their study and improved image reconstruction using a CNN encoder-decoder architecture.
This paper~\citep{kay2008identifying}, used human brain activity from visual areas V1, V2, and V3 as measured by fMRI. They employed linear decoders for decoding the information.
Another approach, as proposed~\citep{qiao2020biggan}, used GANs to decode images. The GAN is provided with Gaussian noise and a category of the image, which is classified by a model based on fMRI brain activity. The generated image is then encoded back to the activity vector by another model and the predicted and true activity vectors are evaluated by an evaluator model. They use fMRI data from the human's brain visual areas V1, V2, V3, and V4. This approach is not suitable, as the categories of images are unknown and the data we use are not fMRI.
Authors of~\citep{kim2021nonlinear} proposed a novel method for decoding visual information from retinal ganglion cells. They used a combination of linear and neural network-based decoders, and a deblurring network to improve the image reconstruction.

Despite the fact, that many researchers study this field, the neural decoding problem remains a significant open challenge. The reason is that the current methods still give poor results in terms of resolving high-frequencies, which describe the fine details in images.

In this thesis, specifically, we aim to mainly address the following questions:

\begin{itemize}
    \item What is the best decoding approach in terms of choice of architecture, and loss function to reconstruct stimuli from the neural activity?
    \item How does the number of neurons recorded or the number of stimuli presented affect results?
    \item Does the intrinsic noise in neural responses represent a substantial limitation to reconstructing visual stimuli?
    \item Besides these questions, we also want to evaluate the influence of different image perturbations on the behavior of different loss functions. 
\end{itemize}


\subsection*{Outline}

In summary, our main focus is to decode the cortical activity and generate stimulus images that accurately represent the visual information processed by the brain. The development of a reliable and accurate model for stimulus image generation has the potential to offer new insights into the visual system and its processing of visual information. By shedding light on the cortical activity in V1 and its role in visual perception, this research may have implications for the development of visual prosthetics, brain-machine interfaces, and virtual reality systems. 
To achieve the stated goals, we utilize the spiking neural model of the V1 cortex developed by Antolik and colleagues~\cite{antolik2018comprehensive}. This model enables us to address questions that would otherwise be hard to answer, such as the impact of the number of neurons and samples in the dataset. Experimental recordings of such large volumes of biological data would be nearly impossible.


The following is the structure of the thesis. Chapter~\ref{methods} describes the proposed models as well as baseline models, loss functions for decoding the stimuli image, and the implementation details. Chapter~\ref{dataset} is dedicated to the description of the datasets we use for training and testing our models. Chapter~\ref{experiments} contains experimental evaluations of the models, and loss functions, but also experiments on the cortical neurons. The final Chapter~\ref{conclusion} comprises of concluding remarks and recommendations for possible future work.